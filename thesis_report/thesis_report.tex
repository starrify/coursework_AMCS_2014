\documentclass[10pt,a4paper]{article}
\usepackage{times}                       % 使用 Times New Roman 字体
\usepackage{CJK,CJKnumb,CJKulem}         % 中文支持宏包
\usepackage{color}                       % 支持彩色

\usepackage{comment}
\usepackage{amsmath}
\usepackage{amssymb}
\usepackage{amsthm}
\usepackage{amscd}
\usepackage{graphicx}
\usepackage{indentfirst}
\usepackage{titlesec}
\usepackage[top=25.4mm, bottom=25.4mm, left=31.7mm, right=32.2mm]{geometry}

%页面设置
\begin{CJK*}{UTF8}{wqyzh}
%\theoremstyle{definition}
%\newtheoremstyle{mythm}{1.5ex plus 1ex minus .2ex}{1.5ex plus 1ex minus .2ex}
%   {\kai}{\parindent}{\wqyzh\bfseries}{}{1em}{}
\newtheoremstyle{mythm}{1ex}{1ex}%定理环境的上下间距.
{\CJKfamily{wqyzh}}{\parindent}{\CJKfamily{wqyzh} \bf}{}{1em}{}%定理内容为宋体, 缩进, 定理名称为黑粗体
\theoremstyle{mythm}%设置定理环境
\newtheorem{thm}{定理~}[section]
\newtheorem{lem}[thm]{引理~}
\newtheorem{pro}[thm]{性质~}
\newtheorem{fact}[thm]{Fact}
\newtheorem{prop}[thm]{命题~}
\newtheorem{ques}[thm]{问题~}
\newtheorem{cor}[thm]{推论~}
\newtheorem{de}[thm]{定义~}
\newtheorem{rem}[thm]{注记~}
\numberwithin{equation}{section}
\end{CJK*}
\renewcommand\refname{\CJKfamily{wqyzh}参考文献}
%\renewcommand{\abstractname}{摘要}
%%%%%%%%%%%%%%%%下面几行用于改变证明环境的定义
\makeatletter
\renewenvironment{proof}[1][\proofname]{\par
\pushQED{\qed}%
\normalfont \topsep6\p@\@plus6\p@ \labelsep1em\relax
\trivlist
\item[\hskip\labelsep\indent
\bfseries #1]\ignorespaces
}{%
\popQED\endtrivlist\@endpefalse
}
\makeatother
%%%%%%%%%%%%%%(http://latex.yo2.cn)
\renewcommand{\proofname}{\CJKfamily{wqyzh}证明}

\renewcommand{\thefootnote}{\fnsymbol{footnote}}
%\titleformat{\section}{\CJKfamily{wqyzh} }{\arabic{section}{1em}{}
\titleformat{\section}{\large \bf \CJKfamily{wqyzh} }{{\bf \thesection\space}}{0pt}{}

\begin{document}
%\setlength{\baselineskip}{1ex}%设置行距
\setlength{\abovedisplayskip}{1ex} %设置公式上边间距
\setlength{\belowdisplayskip}{1ex} %设置公式下边间距
\begin{CJK*}{UTF8}{wqyzh}

\author{Author (ID)}                                 % 作者
\title{计算机应用数学课程论文 \LaTeX 模板}              % 题目
\maketitle                                           % 生成标题

\section{引言}
TEX是由图灵奖得主Knuth编写的计算机程序,用于文章和数学公式的排版。
1977年Knuth开始编写TEX排版系统引擎的时候,
是为了探索当时正开始进入出版工业的数字印刷设备的潜力。他特别希望能因此扭转那种排版质量下降的趋势,
使自己写的{\CJKfamily{wqyzh}书和文章}免受其害。

%\clearpage % 换页,\newpage也可以,推荐\clearpage
我们现在使用的TEX系统是在1982年发布的,1989年又略作改进,增进了
对8字节字符和多语言的支持。TEX以具有优异的稳定性,可以在各种不同
类型的计算机上运行,以及几乎没有错误而著称。


\section{方法概述}
%\CJKfamily{fs}
%中文部分,可以中英文混合
%
%\CJKfamily{wqyzh}
%中文部分,可以中英文混合
%
%\CJKfamily{li}
%中文部分,可以中英文混合
%
%\CJKfamily{kai}
%中文部分,可以中英文混合
%
%\CJKfamily{wqyzh}
%中文部分,可以中英文混合

\section{实验结果}

\section{小结与讨论}

\begin{thebibliography}{MM}
\addtolength{\itemsep}{-0.5em}
\begin{small}
\bibitem{no} text
\end{small}
\end{thebibliography}
\newpage
\end{CJK*}
\end{document}
