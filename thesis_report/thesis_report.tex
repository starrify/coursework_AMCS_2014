\documentclass[10pt,a4paper]{article}
\usepackage{times}                       % 使用 Times New Roman 字体
\usepackage{CJK,CJKnumb,CJKulem}         % 中文支持宏包
\usepackage{color}                       % 支持彩色

\usepackage{comment}
\usepackage{amsmath}
\usepackage{amsfonts}
\usepackage{amssymb}
\usepackage{amsthm}
\usepackage{amscd}
\usepackage{mathtools}
\usepackage{graphicx}
\usepackage{indentfirst}
\usepackage{titlesec}
\usepackage[top=25.4mm, bottom=25.4mm, left=31.7mm, right=32.2mm]{geometry}

%\usepackage{natbib}
\usepackage{chapterbib}
\usepackage{bm}
\usepackage{url}
%\usepackage[]{hyperref}

%页面设置
\begin{CJK*}{UTF8}{wqyzh}
%\theoremstyle{definition}
%\newtheoremstyle{mythm}{1.5ex plus 1ex minus .2ex}{1.5ex plus 1ex minus .2ex}
%   {\kai}{\parindent}{\wqyzh\bfseries}{}{1em}{}
\newtheoremstyle{mythm}{1ex}{1ex}%定理环境的上下间距.
{\CJKfamily{wqyzh}}{\parindent}{\CJKfamily{wqyzh} \bf}{}{1em}{}%定理内容为宋体, 缩进, 定理名称为黑粗体
\theoremstyle{mythm}%设置定理环境
\newtheorem{thm}{定理~}[section]
\newtheorem{lem}[thm]{引理~}
\newtheorem{pro}[thm]{性质~}
\newtheorem{fact}[thm]{Fact}
\newtheorem{prop}[thm]{命题~}
\newtheorem{ques}[thm]{问题~}
\newtheorem{cor}[thm]{推论~}
\newtheorem{de}[thm]{定义~}
\newtheorem{rem}[thm]{注记~}
\numberwithin{equation}{section}
\end{CJK*}
\renewcommand\refname{\CJKfamily{wqyzh}参考文献}
%\renewcommand{\abstractname}{摘要}
%%%%%%%%%%%%%%%%下面几行用于改变证明环境的定义
\makeatletter
\renewenvironment{proof}[1][\proofname]{\par
\pushQED{\qed}%
\normalfont \topsep6\p@\@plus6\p@ \labelsep1em\relax
\trivlist
\item[\hskip\labelsep\indent
\bfseries #1]\ignorespaces
}{%
\popQED\endtrivlist\@endpefalse
}
\makeatother
%%%%%%%%%%%%%%(http://latex.yo2.cn)
\renewcommand{\proofname}{\CJKfamily{wqyzh}证明}

\renewcommand{\thefootnote}{\fnsymbol{footnote}}
%\titleformat{\section}{\CJKfamily{wqyzh} }{\arabic{section}{1em}{}
\titleformat{\section}{\large \bf \CJKfamily{wqyzh} }{{\bf \thesection\space}}{0pt}{}

\DeclarePairedDelimiter\abs{\lvert}{\rvert}%
\DeclarePairedDelimiter\norm{\lVert}{\rVert}%

\makeatletter
\@addtoreset{section}{part}
\makeatother 
%\clearpage % 换页,\newpage也可以,推荐\clearpage

\begin{document}
%\setlength{\baselineskip}{1ex}%设置行距
\setlength{\abovedisplayskip}{1ex} %设置公式上边间距
\setlength{\belowdisplayskip}{1ex} %设置公式下边间距
\begin{CJK*}{UTF8}{wqyzh}

\author{陈鹏宇 (3100000140)}                                 % 作者
\title{关于``用于绘制线条图形的明显脊线''的报告}              % 题目
\maketitle                                           % 生成标题

\section*{METADATA}

\noindent
\begin{tabular}{l}
原文标题:Apparent ridges for line drawing \\
译文标题:用于绘制线条图形的明显脊线 \\
作者:Judd, Tilke, Fr{\'e}do Durand, and Edward Adelson \\
发表于:SIGGRAPH '07 ACM SIGGRAPH 2007 papers, Article No.19 \\
项目主页:\\
\url{http://people.csail.mit.edu/tjudd/apparentridges.html} 
\end{tabular} \\

在如下的报告中我对\cite{Judd:2007:ARLD}进行介绍,包括问题背景、理论基础、实现
方法、结果与分析等。
在问题背景与结果部分带有一些不来自原文的评论。

\emph{注:文中涉及的图表编号与原论文中一致,本报告在分发时应带有原论文的副本,
请去其中查看。另外在项目主页上也可以获得原论文的下载。}


\section{引言}

在非真实感渲染的研究中,本文往往需要使用简单的二维线条来刻画三维物体的形状信息
。一类已知的方法是先将三维物体渲染到图像中,随后使用基于边缘检测的方法来从图像
中得到所需的特征线条\cite{Saito:1990:CRS, Decaudin:1996:CLRS,
Hertzmann:1999:IRSO, Pearson:1985:VCVLDR,Lee:2007:LDAS}。这样的方法由于在图像
空间中进行,往往有精度上的损失,并且通常需要重新采样与向量化以支持进一步的渲染
。

而另一类已知的方法是在物体空间进行计算,往往是寻找物体表面微分几何方面的特殊性
质。一些研究者已经赋予了很多曲线这样的性质\cite{Koenderink:1990:SS,
Ohtake:2004:RVL, Interrante:1995:ETSS}。一个例子是在物体表面曲率极值处取得的脊
线与谷线,这样的曲线刻画了物体表面的重要性质,但视觉上并不是很自然,因这些线条
与视角无关,在物体表面是固定的。一种叫做提示轮廓(Suggestive Contours)的重要
的视角相关曲线最近被\cite{DeCarlo:2003:SCCS,DeCarlo:2004:IRSC}提出。
提示轮廓是闭合轮廓(Occluding Contours)的扩展。这些曲线在径向曲率为零的点处取得。
这些点处于\emph{几乎}是轮廓线的位置上,并且与相邻视角中真实的轮廓线对应。但提
示轮廓只刻画了物体的一部分特性,例如,它会忽略一个立方体上的两面角,以及会完全
忽略图3中构成鼻子的那些脊线。目前并不存在一个单独的方法来包含全部这些需要的曲
线,因此实践中经常讲多种方法结合在一起使用。

本文介绍的新方法可以通过一条单独的方法得到丰富且全面的曲线集合。
这一方法可以被简洁地描述如下:\emph{在物体表面法向量以相对于图像位置的局部最
大变化率变化点处画线}。本文把如上一句所描述的曲线叫做显然脊线。显然脊线的位置依
赖于物体表面的曲率以及投影到观察者平面时的透视缩短。


\section{理论基础}

\subsection{物体表面的脊线与谷线}

给定一个光滑的闭曲面,$\bm{n}(m)$是在点$m$处垂直于曲面指向外的单位法向
量。在$m$点的切平面垂直于这个法向量。直觉上一个曲面的曲率表示一个曲面的弯曲程度,
或是在曲面上的法向量在相邻点之间的变化程度。本文定义在点$m$的曲率算子$S$为
\begin{equation}
    S(\bm{r})=D_{\bm{r}}\bm{n}
\end{equation}
其中$D_{\bm{r}}\bm$为切平面中法向量关于向量$\bm{r}$的方向
导数。$S$,也被叫做是\emph{Weingarten映射}或\emph{形状算子},是一个从$m$点的切平
面到高斯球面上平行于该切平面方向的切向量的线性映射。给定一组基后,$S$可以被表示为
一个对称的$2\times2$矩阵。

对于曲面上每一个点,最大与最小的主曲率$k_1$与$k_2$是$S$的两个特征值,且有
$\abs{k_1}\geq\abs{k_2}$。其对应的特征向量$\bm{e}_1$与
$\bm{e}_2$对应着最大与最小主曲率的方向。

脊线与谷线是主曲率在主曲率方向上取得极值的点的轨迹。它们可以通过
$D_{\bm{e}_1}k_1=0$来描述,其中对于脊线来说$k_1>0$而对于谷线来说$k_1<0$。此外,我
们使用更高阶的导数来保证本文得到的是局部的极值:更高一阶的导数必须是负的
(正的)来保证一条脊线(谷线)所在的点有局部最大(最小)的曲率。

\subsection{视角相关的曲率}

通过将观察投影引入曲率,本文定义了视角相关的曲率。它将导致明显脊线的出现。

\paragraph*{投影}
考虑一个观察者平面$V$,在其上有对一个物体$M\subset\mathbb{R}$的线条绘制。定义一
个平行投影$P$把点集$m\in M$映射到$V'\in V$。如果$m$不是轮廓上的点,则存在一个邻
域使在其上$P$有逆映射$P^{-1}$。

在一个点$m$的$P$的雅可比矩阵$\bar{P}$将$m$点的切向量映射到$V$中的向量。给定切平
面的一组基$(\bm{r}_1,\bm{r}_2)$与屏幕屏幕的一组基$(\bm{s}_1,\bm{s}_2)$,
$\bar{P}$可以被表示为一个$2\times2$矩阵
\begin{equation}
    \bar{P}=
        \begin{pmatrix}
            \bm{r}_1 \cdot \bm{s}_1 & \bm{r}_2 \cdot \bm{s}_1 \\
            \bm{r}_1 \cdot \bm{s}_2 & \bm{r}_2 \cdot \bm{2}_2
        \end{pmatrix}
\end{equation}
注意这里$\bar{P}$在物体表面的每一点都是不同的,并且在除去轮廓点之外的任何地方都
可逆。

\paragraph*{视角相关曲率}
显然,视角相关曲率描述的是从视点看来曲面的弯曲程度。它考虑了物体表面本身的曲率以
及由于曲面朝向带来的透视缩短问题。最终,本文定义在投影平面上的一点$m'$的视角相关
曲率算子$Q$为
\begin{equation}
    Q(\bm{s})=D_{\bm{s}}\bm{n'}
\end{equation}
其中$D_{\bm{s}}\bm{n'}$是物体表面法向量关于投影平面上的向量$\bm{s}$的方向导数。
在$m'$点的视角相关曲率$Q$是一个从投影平面到高斯球面上的切向量的线性映射。注意这
里求导是关于投影平面进行的,但对法向量的改变是在物体空间中。本文\emph{不}把法向
量投影到投影平面中,因这里动机之一是基于在物体空间中法向量的阴影,而将法向量投影
到投影平面中会潜在地忽略一些沿着视觉方向的光线分量。

随后使用链式法则来获得$Q$基于表面曲率的表示,
\begin{equation}
    \boxed{Q=S\bar{P}^{-1}}
\end{equation}。
其中在切平面内选取的用来表示$S$与$\bar{P}$的基是同样的。$Q(\bm{s})$是切平面中的
一个向量,描述沿着投影空间中$\bm{s}$方向移动时曲面法向量的变化情况。

本文关心的是这一变化情况的\emph{大小}的极值,所以本文定义最大视角相关曲率为
\begin{equation}
    q_1=\max_{\norm{\bm{r}}=1}\norm{Q(\bm{r})}
\end{equation}
其中$\bm{r}$是切平面中的一个向量。这是物体表面一点在一个椭圆方向上的各向曲率的
模长的最大值(见图4)。

视角相关曲率将对视角的依赖添加到传统的曲率定义中去。在正对着视觉平面的部分,曲率
与视角相关曲率是相同的(见图5)。而当物体的一部分转向远离视觉平面的方向时,视角相
关曲率会大幅增加,并且视角相关曲率的主曲率方向会向视向量靠近。

\subsection{明显脊线}

参考脊线的定义,本文定义明显脊线为在视角相关曲率的主曲率方向$\bm{t}_1$的方向上
取得最大视角相关曲率$q_1$的局部最大值的点的轨迹。这发生在
\begin{equation}
    D_{\bm{t}_1}q_1=0
\end{equation}
时。通过选取那些更高阶导数是负的点,本文只保留$q_1$取得局部最大值的点。注意视角相
关曲率永远是非负的,本文对明显脊线的定义事实上刻画了类似脊线与类似谷线的特征。例
如,人脸模型中关于鼻子轮廓的线条与两片嘴唇中间的线条分别是脊线与谷线(见图1)。我
们可以通过\emph{物体空间}的曲率来区分这两种特征:类似脊线的特征有$k_1>0$而类似
谷线的特征有$k_1<0$。


\section{网格上的明显脊线计算}

本文将视角相关曲率的计算推广到离散的三角形网格上去:

\paragraph*{计算视角相关曲率}
本文利用标准的技术来在曲面网格的每一个点估计曲率
$S$\cite{Rusinkiewicz:2004:ECD}。随后本文乘上投影变换的逆变换来得到
$Q=SP{^-1}$。尽管本文的推导依赖于平行投影,实践中本文也可以将一个透视投影的相机
在每一个顶点近似地取为一个局部的平行投影$P$。更具体地,本文将每个顶点的投影射线
理解为视点与顶点之间的连线。从$Q$本文可以进一步地计算在每个顶点最大视觉相关曲率
$q_1$以及对应的最大视觉相关曲率的主方向$\bm{t}_1$。

\paragraph*{计算视觉相关曲率的导数}
本文使用有限差分法来估计$D_{\bm{t}_1}q_1$(见图6(a))。为了计算在网格顶点
$p$的导数,本文在与$p$点在$\bm{t}$方向上相邻的三角形的两个顶点$w$与$w'$上计算视
角相关的曲率,并将$p$与两个$w$点之间的有限差分取平均。对于在一条边上的点$w$,其视
角相关的曲率是通过线性插值在相邻的两个顶点$u$与$v$的视角相关曲率得到。

\paragraph*{寻找一个连续的$\bm{t}_1$场}
注意$\bm{t}_1$是一个从顶点出发指向两个相反的方向之一的向量(见图6(b))。为使这个
向量场在网格表面连续,本文将$\bm{t}_1$反转使其指向正的微分方向,即视角相关的曲率
增长的方向。

\paragraph*{定位跨越零点的部分}
本文使用一个受到\cite{Ohtake:2004:RVL}启发的方法来定位网格上视角相关曲率跨越
零点的部分(见图6(c)与(d))。如果$\bm{t}_1$在同一条边的两个顶点上指向同样的方向,
那么这条边上就没有视角相关曲率导数为零的部分。而如果它们指向不同的方向(夹角超过90度)
那么这条边上就有零点。随后本文使用顶点上的导数值来进行插值得到零点的位置
\cite{Hertzmann:2000:ISS}。

\paragraph*{修剪}
使用如上一段的方法定位到的零点包含了法向量变化量的局部最大与最小值,但本文只需
要在最大值处画线。为此本文从每个顶点向零点所在的线引一条垂线(见图6(c)与
(d))。若$\bm{t}_1$与垂线夹角为锐角则被测试的零点是局部最大值。否则该零点就被剔除
。这样的测试方式比\cite{Ohtake:2004:RVL}中建议的更鲁棒,因在这篇论文中作者使用
了网格的边来近似垂线。

\paragraph*{阈值}
在局部最小值点被剔除后,仍然有很多线条留下。这是因为本文的方法找到的是全部的局部
最大值,与原本的视角无关曲率高低无关。显然本文不止希望在选中的点处视角无关曲率是
局部最大值,本文还希望这个局部最大值足够大。因此本文设定了一个阈值来剔除曲率最大
值不足这一阈值的线条。本文用网格的特征尺寸(平均边长)来缩放这个阈值使其变得与维
度无关。全部的特征线条的定义都需要一个类似的阈值。



\section{实验结果}

\paragraph*{性能评估}
脊线与谷线以及提示轮廓可以被很快地计算出来,因曲率是物体本身的性质,在视角变换下
不变,可以被在渲染之前就预计算好。明显脊线依赖视角相关曲率以及它的导数,因此必须
被在每一帧重新计算。使用本文在2.33 GHz Intel Core 2 Duo处理器的Macintosh机器上
的未经优化的代码,小型物体的显然脊线可以实时被计算,对于约50000个多边形的网格,渲
染一帧的时间为约1.5秒,对于月250000个多边形的网格,渲染一帧的时间为约9秒。

作为补充,性能之外的另一个局限是由于明显脊线的计算需要引入高阶导数,这使它易于受
到网格本身的噪声的影响。

\paragraph*{渲染质量评估}
现在本文在渲染质量上对明显脊线与其他主要的特征线条进行比较。公平起见,本文选用常
数作为线条宽度,并且用锋利切断的方式处理线条的两端,以及用灰度的像素个数来作为线
条生成中会用到的阈值。

\paragraph*{侧影与轮廓线}
轮廓线位于法向量与视角方向垂直的点。当位置向轮廓线趋近时视角相关的曲率会由于投
影的关系趋向于增长到无穷大(图5)。尽管视角相关的曲率在轮廓线处是奇异的,本文在网
格上的计算方法的确能够提取出轮廓线。本文中出现的明显脊线效果图都没有经过传统方
式的边缘提取。这一点很重要,因为这样明显脊线就可以独立工作。与此同时其他的特征线
条的结果都必须与轮廓线结合使用。

\paragraph*{脊线与谷线}
明显脊线与传统的脊线关系很密切:他们的定义是基本一致的,区别在于前者引入了投影
的因素。通过引入投影的因素,明显曲线在视觉上较传统的脊线与谷线更贴近于物体本身的
效果。当投影的因素影响很小,例如在正对着视点的部分,传统的脊线与谷线和明显脊线很
相似。当这些部分远离视点的时候,明显脊线与传统的脊线会变得更加不同。

在脊线与谷线表现得很好的一些例子中,例如摇臂(图7)与石柱的底部(图12)这样的刚体,
明显脊线成功地模仿了同样的细节。在脊线与谷线表现得不那么好的例子中,明显脊线会在
视觉上更加让人满意。从桌布的例子(图8)中看,脊线捕捉了桌子边缘这一重要信息,而明显
脊线则提供了一个更加平滑的方式,使桌子边缘两端的部分消失了。同样的表现发生在平滑
的立方体与十二面体的例子中(图10)。脊线捕捉任意的局部最大化的褶皱,但这些信息在视
觉上重要性很低,与此同时明显脊线将这些线条合并进了附近的更加重要的轮廓线中。由于
脊线与谷线可以从物体本身的性质中得到,他们的出现有时会像物体表面的人工标记一样,
并在Bust(图1)与Max Planck(图9)的鼻子与嘴部产生盒子一样的效果。对于这些人形的模
型,明显脊线则会生成更加有吸引力的线条。

明显脊线在一些使脊线有着病态定义的环境下仍是良定义的。在一个对称的高斯突起
(Symmetric Gaussian Bump)(图13)上,明显脊线将轮廓线延伸了一段,但脊线在这个例子
中并不存在,因在那个物体上并不存在曲率的最大值点。

\paragraph*{提示轮廓}
明显脊线与提示轮廓是根本上不同的两种特征线条。从直觉上看,提示轮廓关注
\emph{法向量}的极值,而明显脊线关注的是\emph{法向量的变化量}的极值。提示轮廓从
\emph{视向量}的角度关注曲率,而明显脊线从$\bm{t}_1$即\emph{从视平面看来法向量变
化率最大的方向}。这些方向的定义不同,但一些情况下他们的确是一致的:透视上的缩短
沿着视向量的方向发生,因此也使得视角相关的曲率在那个方向上发生了膨胀,进而使其更
有可能成为$\bm{t}_1$的方向。然而即使这些方向相同,提示轮廓要寻找的是这些方向上曲
率为\emph{零}的点,而明显脊线寻找的是曲率的\emph{最大值}。

既然提示轮廓与明显脊线有如此多的不同,那么很难说究竟哪一个明显地更优。两者都是很
有趣的选择,并提供了不同的优点与弱点。

提示轮廓与明显脊线在绘制单层或是双层线条的场景上是交换的。在大脑模型的例子中(图
11),提示轮廓在裂缝处绘制双层线条,而此处明显脊线给出了更简洁的单层线条。本文的明
显脊线定位到了裂缝的谷,而提示轮廓定位到了在谷两旁的弯曲地带。然而在柱子模型中
(图12),本文在斜角处给出了双层线条(脊与谷各一),而提示轮廓给出了更简单的单层线条
(在弯曲地带)。这一点在柱子的中部也可被观察到。使用单层线条描绘弯曲的部分对于复杂
模型来说是好的选择,但对简单的模型而言可能简化太多。在加入了阴影的摇臂模型中
(图7),提示轮廓忽略了其表面与轴的交界处的一条重要曲线,与明显脊线的结果对比起来
就显得缺失了特征。

提示轮廓的一个很强的特性是它们会扩展轮廓线。明显脊线看上去也做了同样的事情(图
13)。明显脊线将轮廓线扩展边长,因临近的区域的透视缩进效应很高,导致了很高的视角相
关曲率值。注意摇臂模型中(图7),明显脊线将轮廓线扩展到类似边缘的结构上,而提示轮廓
将轮廓线落栈到表面上。

一些凸区域的重要特征并没有被提示轮廓刻画出来。例如在桌子的边缘上(图8)与圆角立方
体的中部以及十二面体上(图10)的线条并没有被绘制出来,而明显脊线在如上的模型中成
功地刻画了重要的特征。

\paragraph*{与边缘检测的比较}

如果使用在视点的点光源以及Lambertian光照模型,提示轮廓会被绘制在物体有阴影的部
分。就如同在桌布的例子中(图8),深色的阴影区域与提示轮廓是一致的。这种对应关系只在
特定的光照模型下有意义,如果光源被移除,提示轮廓的线条会看上去是任意绘制的。另一
方面,本文最初的试验显示明显脊线明显脊线的线条被绘制在多个不同位置的光源产生的
阴影的平均位置上。这也说明了明显脊线选择绘制线条的位置与光源位置是无关的。

在图14中本文展示了一个蒙特卡洛试验,其中一个漫反射的表面被使用数千个随机的光照
配置照射,并被从一个固定的视点观察。对这数千张图片使用Canny边缘检测算子
\cite{Canny:1986:CAED}的输出求平均得到的结果与明显脊线的线条绘制结果显著地一致
。本文使用闪光灯对现实的物体模型进行光照并拍照,将得到的照片应用于Canny边缘检测
算子并与使用明显脊线绘制的该物体三维扫描的模型的特征曲线进行了比较(图15)。尽管
两者的视点并没有完美地对应,本文仍可看出两种特征线条提取方法的结果是一致的,说明
本文的线条与真实物体阴影的边缘是相关的。这也体现了与\cite{Raskar:2007:NPRC}的
有趣关系。


\section{小结与讨论}

本文引入了\emph{明显脊线}用于非真实感的线条绘制。明显脊线给出了视觉效果很好的线
条绘制,并能刻画关于物体形状的重要信息。一系列的线条类型,例如轮廓线与二十面体的
锋利脊线,都是本文的定义的特殊情况。在传统的脊线与谷线表现很好的情况下,明显脊线
也在同样的位置表现很好。当脊线与谷线表现得生硬或太过方正时,明显脊线将其修改使得
感性上更加合适。明显曲线与提示轮廓有关,但有区别。两者都是基于视角相关以及视角无
关的属性。两者都能给出令人满意的结果,但在很多情况下本文发现明显脊线的图像在表现
上更加吸引人。明显脊线对于人脸的特征给出更自然的输出图像,对于凸表面给出更有价值
的信息。最初的边缘检测实验同样说明明显脊线是与特定的光照环境无关的重要线条。

\end{CJK*}
\end{document}
